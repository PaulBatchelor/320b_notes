Notes for vid9.mp4

\subsection*{Bode plot of simplest electrical LPF}

The bode plot applies to the amplitude response (magnitude frequency response).

In the s-plane:
$G(\omega) \stackrel{\Delta}{=} \vert H(j \omega)\vert
=\vert \frac{1/\tau}{j\omega + 1/\tau} \vert =
\vert \frac{\sigma}{j \omega + \sigma} \vert 
$

We will plot the magnitude response over a log frequency scale in DB.

When $\omega = 0$, the gain is 1.

0 db for low frequencies, until $\omega$ becomes comparable to $\frac{1}{\tau}$

When we reach $\sigma$, what is our gain? $1/\sqrt{2}$, which is -3db. 

For $\omega > \sigma$, it is 6db/octave rolloff. Our bode plot looks like a 
half decent lowpass filter! It's got a nice pass band, and a well dfined -3db 
point. This is cascadable, so they can be stacked to get a steeper rolloff.

\paulhint{frame 13a taken at 3:26}

\subsection*{Pole zero diagram}
Recall the pole-zero plot:

What is the bandwidth of that pole? 

Bandwidth is interval between -3db points. We just derived that $\sigma = -3d$,
so we can just rotate to the $j\omega$ axis, which is why the s-plane
more enjoyable than the z-plane. 

You can say that in the z-plane your 
frequency axis is the unit circle, and if you straighten it out you'll
get the s-plane. The $j\omega$ axis is a great circle in the extended
complex plane; if you put everything on a sphere, then the $j\omega$ axis
is a circle. 
