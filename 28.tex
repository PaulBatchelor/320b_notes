Video notes: vid16.mp4
\subsection*{Cepstrum}

\begin{itemize}
\item{Derived from \textit{Spectrum} with the first syllable flipped}
\item{Quefrency is the same thing}
\item{Complex Cepstrum: $
\stackrel{\sim}{h}(n) \stackrel{\Delta}{=} \mbox{IDFT}_n [ln H(e^{j\omega T})]
$}
\item{
$
\stackrel{\sim}{h}(n) \stackrel{\Delta}{=} \mbox{IFT}_n [ln 
\vert H(e^{j\omega T}) \vert ]
$
}
\item{Assume $h$ is real of length N or infinity, which means that
$h(n)$ is real (and possibly inifite)}, in addition to causal (negative times)
\item{
Cepstrum can make non-causal things and make causal in a very convenient way.
}
\item{
We are going to convert non-minimum phase spectrum to minimum phase by taking
the non-causal components and folding them aroudn
}
\item{
$H(e^{j\omega T}$ is Hermitian. If you flip it, you get the conjugate of H.
This also means that magnitude is even in $\omega$, and the phase
of the spectrum is odd. 
}
\item{
$ln \vert H \vert$ is even. 
}
\item{
log of complex spectrum: 
$ln H = ln \vert H \vert + j + \angle H
$
Where $H = \vert H \vert \cdot e^{j \angle H} $
The real part is even, the imaginary part is odd. Look at your fourier theorems
on that!
}
\item{
$
\stackrel{\sim}{h}(n) = \stackrel{\sim}{h}_r(n) + \stackrel{\sim}{h}_o(n) 
$
}
\end{itemize}
