Class notes: 2-23-16

\subsection*{Symmetric Linear phase Filters, and other questions}

\begin{itemize}
    \item{A question related to problem 1 of the hw. on page 225}
    \item{
        \textbf{What is the $\pi$ for?} I think it has to contribute to 
wrapping? Don't take my word for it I missed it.
    }
    \item{
        Question about allpass filters: \textbf{Reciprocal conjugate in 
        previous homeworks, in the book we raise
        epsilon to the negative one.  What happened to the conjugate?} When you 
        take the reciprocal, you get the conjugate.  this is okay if they are real, 
        because they just swap places. Yes, it's a special case, but it's a special 
        case that we always do.
    }
    \item{
        \textbf{Constant modulus algorithm:} rejects the noise in such away to push the modulus to 1. 
        it's a form a noise removal. Used in phase vocoders sometimes. 
    }
    \item{ 
        \textbf{How do you go about making a separation table? 
        Is it the same thing as a tuning table?} 
        This comes up talking about a the moogvcf. This comes up in the paper by Tim Stilson.
        Apparently, they are two different things? Read up on this. 
        This comes down to frequency warping in the BLT. We normally reduce these things 
        (related to trig) in a table.
    }
    \item{ The extra delay we add in the transfer function messes up the orthagonality. The
     separation table is a way of fixing this. }
    \item{ If you're using the BLT: you will get some instabilities aroudn some of the frequencies.  }
    \item{ The moogvcf is an example of cases where the BLT isn't necessarily the best approach. 
        Other approache could be backward euler with oversampling. 
    }
    \item{
        What's really critical in a feedback system is the \textit{phase margin}.
    }
    \item{
        You want to look at the overall stability with or without the delay. 
    }
\end{itemize}

\subsection*{Moogvcf in FAUST: Overview, and more questions}
\begin{itemize}
\item{
moogvcf is defined is it's natural topology
}
\item{ 
sent into 4 one pole filters
}
\item{
uses BLT
}
\item{
after the string of polls, we have our feedback
}
\item{
Corner freq, in Hz
}
\item{
The s family of functions use the BLT internally, which leads 
to stability problems. This is fine for lower frequencies. 
}
\item{
New improved moogvcf: first one julius learned about. Exactly the same way (follows natural topology)
except for way poles are digitized: Implemneted as a unity gain smoother. 
}
\item{
K can go all the way up to 3.8, where 4 is the stability limit. The mapping between K and Q is 
very significantly changed.
}
\item{
When you see the FFT, you can see the the Q melts down after each one. Correction table
fixes this. 
}
\item{
With memory these days, you can practically implement algorithms that require infinite memory. 
}
\item{
Think about massive parrallelism and enormous amounts of memory 
(although they don't go together well). 
}
\item{
In the old days: there were array processors: you'd ship your arrays to the array processors, 
and they'd calculate blazingly fast. Even though the computation points are fast, the IO is slow.
We just need NVdia to care about audio! 
}
\item{
With Q table, you fix the Q to have the same height and width... do you need a separate
gain table to set the peak gain? 
}
\item{
Speech modelling is always done with allpole filters. 
}
\item{
For constant Q: you will have set declines: no matter how you correct the Q it will stay there.
This might be on the final...
}
\item{
New Section Preview: Cepstrum, digitizing vocal tract
}
\item{

}
\item{

}
\item{}
\item{}
\item{}
\end{itemize}
