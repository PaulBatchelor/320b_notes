Video notes: vid15.mp4
\subsection*{Allpass filters (review)}

\begin{itemize}
\item{
z-plane: $\vert H(e^{-j\omega T}) \vert = 1$ and
s-plane: $\vert H(j\omega) \vert = 1$ 
}
\item{
z-plane: $\frac{z^{-N}A(z^{-1})}{A(z)}$
}
\item{
s-plane: $\frac{A(-s)}{A(s)}$
\paulhint{It would be a good idea to have doodles of what the poles
look like}
}
\end{itemize}

\subsection*{Min Phase Allpass Decomposition }
\begin{itemize}
\item{MP assumes stable poles}
\item{Zero outside the unit circle is not MP}
\item{In MP+allpass decomp: take that zero, reflect it inside}
\item{
We're taking the original filter which has a non-mp zero, and 
expressing it as a mp filter where the pole and zero are inside 
the unit circle, times an allpass filter with the non-MP zero and
a pole at the inverse. 
}
\item{
We can write the decomposition as
$
\frac{1 - q_e z^{-1}}{1 - p_i z^{-1}} =
\frac{1 - q_i z^{-1}}{1 - p_i z^{-1}} \cdot
\frac{1 - q_ez^{-1}}{1 - q_i z^{-1}}
$, Where $q_i = \frac{1}{q_e}$
}
\item{
MP in s-plane is poles and zeros in the LHP
}
\item{
What would the bilinear transform do to this? Only have to remember
one case, and remember that case under the BLT. "I freely use the BLT
for z-s and s-z"
}
\item{
There are a lot of times when you need to know about MP filters, espeically
for filter design. If you can convert the desired frequency response
to a MP filter, you are going to get much nicer results in your 
filter design.
}
\item{
If the filter design is sensitive at all, you can really make it fall
flat on it's face if you try to make it do a non-MP filter design.
}
\end{itemize}
\subsection*{Geometric Series Revisitied}
\begin{itemize}
\item{Recall: 
$\frac{1}{1-R} = 1 + R + R^2 + \cdots \infty $ when $\vert R \vert < 1$
}
\item{
You can take this farther: suppose $\vert R \vert > 1$, then rewrite as:\\
$\frac{1}{1 - R} = \frac{-R^{-1}}{1 - R^{-1}} =
-R^{-1} [1 + R^{-1} + R^{-2} + \cdots ] < \infty
$, When $\vert R \vert > 1$
}
\end{itemize}
\subsection*{Apply to one-pole filter}
\begin{itemize}
\item{
$H(z) =  \frac{1}{1 - pz^{-1}} = 1 + pz^{-1} + p^2z^{-2} + \cdots$
}
\item{
This is just a z transform of the impulse response of a one-pole filter. A
causal, stable, filter, where $\vert p \vert < 1$
}
\item{
Requires that $\vert pz^{-1} \vert < 1$, Which implies
$\vert p \vert < 1$
}
\item{
    \josquote{You know you're really done when you read through your 
    latest draft, and you don't want to touch it. If you can 
    get to a fixed point in your own brain, then you are ready to publish.
    That's what you submit. Don't submit it until you get to your internal 
    fixed point. People don't have time to look at things that 
    aren't perfect anymore. That's the dividend of the internet. The internet
    makes it possible for one person to do it right and nobody else
    has to do it. 
    }
}
\item{
Go back to the time domain: we normally think of this as the inverse
z-transform. We recognize this as the z-transform of a causal 
exponentionally decaying sequence. 
}
\item{
If you wanted to do a true inverse transform: make it an inverse DTFT 
(IDTFT) to get $h = [1, p, p^2, \cdot]$ or $h(n) = u(n) p^{n}$, 
where $u$ is a unit-step function.
}
\item{
We are implicitly assuming that the Z-transform has a region 
of convergence that includes the $j\omega$ axis or the $e^{j\omega T}$ axis.
It must be in the region of convergence of the Z-transform.
}
\item{
By assuming that, it means that poles inside the unit circle
correspond to causal decaying exponentials.
}
\item{
By that same assumption, poles outside the unit cirlce correspond
to anti-causal exponentials that decay to left, that decay to $-\infty$.
}
\item{
This is a new concept: what we were doing before is inisiting that 
all poles be inside the unit cirlce OR if we had poles outside the unit
circles, the region of convergence would be greater than that. 
}
\item{
Now, we're letting the pole cross that circle, and now the exponential
flips around and points the other way now.
}
\end{itemize}
\subsection* {Rewritten case , $\vert p \vert > 1$ 
($pz^{-1} > 1$)}
\begin{itemize}
\item{$\frac{1}{1 - pz^{-1}}=
\frac{-p^{-1}z}{1 - p^{-1}z}
= -p^{-1}z [ 1 + p^{-1}z + p^{-2}z^2 + \cdots]  \\
= -[p^{-1}z + p^{-2}z^2 + p^{-3}z^3 + \cdots]
$
}
\item{
IDFT $= -u(-n - 1)p^{n}$
}
\item{
\textbf{Example}: $H(z) = 
\frac{1}{1 - \frac{1}{2}z^{-1}} +
\frac{1}{1 - 2z^{-1}}
$
}
\item{
First term: stable pole at $z = \frac{1}{2}$,
Second term: unstable pole at $z = 2$,
}
\item{
$H(z) = \frac{1}{1 - \frac{1}{2} z^{-1}}
+ (
\frac{-\frac{1}{2} z^{-1}}{
1 - \frac{1}{2}z^{-1}
}
)$
}
\item{
\paulhint{There's about 5 more minutes of this video, worth watch the last 10 minutes}
}
\end{itemize}

