% ============================
% Sat Jan  9 14:58:47 PST 2016
% ============================

% NOTE: video watching not done yet

\subsection*{Simplest Recursive Lowpass Filter}

$y_n = x(n) + Py(n-1)$

Where $0 < P < 1, \\ 
n = 0, 1, 2, \dots
y(-1) \stackrel{\Delta}{=} 0
$\\\\
Z transform:\\\\
$Y(z) = X(z) + Pz^{-1}Y(z)$

$=> H(z) \stackrel{\Delta}{=} \frac{Y(Z)}{X(Z)}
= \frac{1}{1 - Px'}
$
\\

\josquote{Ask about filters: what is their maximum gain?}


\subsection*{Max Gain}

Max gain is defined as:

$\frac{1}{1 - R_{Max}}$

Rmax = maximum Poles modulai  %max\{|Pi|\}

Maximum gain is the one-pole, whose closest to thunit circle.


\subsection*{Time Constant of One Pole}

$e^{-t/\tau}$ =
$e^{-nt/\tau}$  =
$[e^{-t/\tau}]^n$ 

$
P = e^{-t/\tau}
= 1- \frac{T}{\tau} %approx
= \frac{1}{1 - P} %approx
$\\\\

$
\frac{1}{1 - |P|}
$ 
= for any pole $p$ in C is approx equal to time constant in sample.
which is equal to the peak gain.


also,
$
\frac{1}{1 - |P|} =
$ 
Peak gain.

$
H(z) = \frac{1}{1 - PZ^{-1}} =
$ 

$h = [1, p, p^2, p^3, \dots]$\\
$h(n) = p^n$

\subsection*{Stability}

$h(n) -> 0$

%$P f E: H(z) = \sum

\josquote{A fliter is stable if |Pi| <1 upsidedownAi = 1, \dots, N}



\subsection*{Pole BW}

Bandwidth (BW) \\

Bandwidth is defined is -3db points to the left and right of 0db point. 
Distance from left and right points is the bandwidth.

Bandwidth is in Hz. 

-6db is a power of 2, (dividing by 2), -3db is dividing by $\sqrt{2}$

What do we do to estimate it from the pole?
\\
Estimate the -3db points. 

$|P| = e^{-\pi B T}$
where $B$ = bandwidth in Hz.

The derivation assumes that you want a narrow bandpass.

ANother way:

$B = \frac{-ln(|p|)}{\pi T}$


\subsection*{S-plane poles and zeros}
$h(t) = e^{-t/\tau}$

The generalized transform of this is the laplace transform.

\subsection*{Laplace transform}
Go back to this: (approx 30min. vid4)\\
Look this up:\\
H(s) = L h etc... etc...

$\int_0^{\infty} e^{-t/\tau}e^{-st}dt$
$= \int_0^{\infty} e^{-t/\tau}e^{-st}dt$ \\\\

$e^{pt} \leftrightarrow \frac{1}{s - p}$
\\
more commonly written as: \\
$e^{\alpha t} \leftrightarrow \frac{1}{s - \alpha}$\\
\josquote{this is how you get poles in the s-plane}

Let $\alpha = \sigma_0 + j\omega_0$

I stopped here (towards the end of video... get back to this)



