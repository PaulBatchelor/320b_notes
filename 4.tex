% ============================
% Sat Jan  9 14:58:47 PST 2016
% ============================

% NOTE: video watching not done yet
Video notes: 
\subsection*{Simplest Recursive (Lowpass) Filter}

$y_n = x(n) + Py(n-1)$

Where $0 < P < 1, \\ 
n = 0, 1, 2, \dots
y(-1) \stackrel{\Delta}{=} 0
$\\\\
\textbf{Z transform:}\\\\
$Y(z) = X(z) + Pz^{-1}Y(z)$

$=> H(z) \stackrel{\Delta}{=} \frac{Y(Z)}{X(Z)}
= \frac{1}{1 - Px'}
$
\\

If you keep P positive, it is a some kind of lowpass.

\josquote{Ask about filters: what is their maximum gain?}

Now, where does that thing have a Pole? When $Z = P$, it blows up, so that
is where the pole is. The P is always going to be real, for our purposes.



\subsection*{Max Gain}
\begin{itemize}
\item{
Max gain is defined as:
$\frac{1}{1 - R_{Max}}$
}
\item{Rmax = maximum Poles modulai}
\item{Maximum gain is the one-pole, whose closest to the unit circle.}
\item{
Recall: PFE: $H(z)$ can be expressed as a sum:
$H(z) = \sum\limits_{i = 1}^{N} \frac{r_i}{1 - P_i z^{-1}}$
Assuming ($N>M$, and $P_i$ is distinct)
}
\item{
If $M > N$, then you'll get some FIR filter and you'll end up doing long 
division.
}
\end{itemize}


\subsection*{Time Constant of One Pole}
\begin{itemize}
\item{Time constant is a measure of duration.}
\item{
$\tau$ is the time to decay to $1/e$. 
}
\item{
T60 is approximately 6.97 time constants.
}
\item{Defined relative to the canonical form of the exponential: \\
$e^{-t/\tau} \rightarrow $ 
$e^{-nt/\tau}$  =
$[e^{-t/\tau}]^n$ 

}
\item{
$P^n$ is the sampled exponential, which you can expand into it's
taylor/mclauren series. 
\begin{align*}
P &= e^{-T/\tau} \\
&= 1- \frac{T}{\tau} + 
(\frac{T}{\tau})^2 \frac{1}{2} + 
(\frac{T}{\tau})^3 \frac{1}{3!} + \cdots \\
&\approx 1 - \frac{T}{\tau}
\end{align*}
}
\item{
If you assume that tau is much greater than the sampling interval (pretty normal),
then you can ignore the higher-order terms, which allows us to do the 
approximation above.
}
\item{
We can now solve for $\tau$, using our truncated taylor series approximated 
above. $\tau/T$ is the time constant in samples. which is approximately 
$\frac{1}{1 - P}$.
}
\item{
We know by the graphical method that if you take any pole P and look at 
the gain at the pole angle you get $\frac{1}{1 - \vert P \vert}$ because
that is the distance of the pole to the unit circle. This is also the maximum
gain. 
}
\item{
If you think of an exponentially weighted moving average, it's basically
saying that $\tau$ is the effective duration of it. 
}
\item{
The integration of $e^{-t/\tau}$ is $\tau$.
}
\item{
$
\frac{1}{1 - |P|}
$ 
= for any pole $p$ in C is approx equal to time constant in sample.
which is also equal to the peak gain of the pole.
}
\item{
Impulse response of Pole: $H(z) = \frac{1}{1 - pz^{-1}}$
What is the inverse z transform of this? we can kind of guess.

$h = [1, p, p^2, p^3, \cdots]$

It's a sampled exponential where the samples are just powers of the pole,
so $p^n$ is the impulse response: $h(n) = p^n$. True for any pole, but we use
it primarily for real poles.

This exponential decay is our impulse response, it's our exponentially 
weighted moving average impulse response.
}
\item{
$\frac{1}{1 - \vert p \vert}$ what you get when you add the impulse response 
up. It's the effective duration of that thing and it's the effective sum of the
thing.
}
\end{itemize}


\subsection*{Stability}
\begin{itemize}
\item{We are always going to want our filters to be stable.}

\item{PFE: $H(z) = \sum\frac{r_i}{1 - pz^{-1}}$ sum of one-poles. 

$h(n) = \sum\limits_{i = 1}{N} r_i p_i^n$
}
\item{
$h_i = [1, p_i, p_i^2, \dots] * r_i$
This must decay, so p needs to be less than 1.
}
\item{
A filter is stable if and only if (iff) $\vert p_i\vert  < 1$ for all i. 
}
\item{
Poles must be strictly inside the unit circle. 
}
\item{
We use a lot of marginally stable filters to make oscillators. Many of these
are implemented in Faust. 
}
% $h(n) \rightarrow 0$

%$P f E: H(z) = \sum

\josquote{A fliter is stable if |Pi| <1 upsidedownAi = 1, \dots, N}
\end{itemize}


\subsection*{Pole Bandwidth (BW)}
\begin{itemize}
\item{What is bandwidth? Passband is generally defined as the integeral
between the -3db to the left and right of 0db point, that is what we call
bandwidth.}
\item{
-6db is dividing by 2. -3db is dividing $\sqrt{2}$, the "half power point".
}
\item{
What do we do to estimate it from a pole? We know that when we fly by a pole
on the unit circle we get this nice localized gain. We just have to estimate
the -3db points, and that will give us the bandwidth.
}
\item{
Formula
\begin{align*}
\vert P \vert \approx e^{-\pi B T}
\end{align*}
Where $B$ is bandwidth (in Hz). 
}
\item{
The derivation of this equation that you want a resonably narrow bandpass. 
Reasonably resonant. 
}
\item{
Equivalent:
\begin{align*}
B = - \frac{\ln \vert P \vert}{\pi T}
\end{align*}
}
\item{
Normally, this is were it would be proved. But things are much simpler
in continuous time, so we will analyze the s-plane case 
$H(s) = \frac{-\sigma_p}{s - p}$. 
}
\end{itemize}
%Bandwidth (BW) \\
%
%Bandwidth is defined is -3db points to the left and right of 0db point. 
%Distance from left and right points is the bandwidth.
%
%Bandwidth is in Hz. 
%
%-6db is a power of 2, (dividing by 2), -3db is dividing by $\sqrt{2}$
%
%What do we do to estimate it from the pole?
%\\
%Estimate the -3db points. 
%
%$|P| = e^{-\pi B T}$
%where $B$ = bandwidth in Hz.
%
%The derivation assumes that you want a narrow bandpass.
%
%ANother way:
%
%$B = \frac{-ln(|p|)}{\pi T}$
%
%
%\subsection*{S-plane poles and zeros}
%$h(t) = e^{-t/\tau}$

% The generalized transform of this is the laplace transform.

\subsection*{S-plane poles and zeros}
\begin{itemize}
\item{Consier the basic exponential $h(t) = e^{-t/\tau}$}
\item{
\begin{align*}
e^{pt} \leftrightarrow \frac{1}{s - p}
\end{align*}
}
\end{itemize}

% also,
% $
% \frac{1}{1 - |P|} =
% $ 
% Peak gain.
% 
% $
% H(z) = \frac{1}{1 - PZ^{-1}} =
% $ 
% 
% $h = [1, p, p^2, p^3, \dots]$\\
% $h(n) = p^n$

\subsection*{Laplace transform}
% Go back to this: (approx 30min. vid4)\\
% Look this up:\\
% H(s) = L h etc... etc...
% 
% $\int_0^{\infty} e^{-t/\tau}e^{-st}dt$
% $= \int_0^{\infty} e^{-t/\tau}e^{-st}dt$ \\\\
% 
% $e^{pt} \leftrightarrow \frac{1}{s - p}$
% \\
% more commonly written as: \\
% $e^{\alpha t} \leftrightarrow \frac{1}{s - \alpha}$\\
% \josquote{this is how you get poles in the s-plane}
% 
% Let $\alpha = \sigma_0 + j\omega_0$
% 
% I stopped here (towards the end of video... get back to this)



