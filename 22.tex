Class notes: 2-16-16

\begin{itemize}
\item{One pole at DC: what can you do with one pole?}
\item{With multiple poles at DC, you can do anything}
\item{What is the transfer function? $H(s) = \frac{1}{s}$}
\item{IR in the time domain? $h(t)e^{-st}dt$}
\item{Integration theorem}
\item{What do you get when you integrate an impulse?}
\item{Unit step impulse response}
\item{Model: realization} % What is this?
\item{What, when you kick it, just steps (a unit step response)? A switch... no not a LTI environment}
\item{Capacitor: Q = cv... the charge of the capacitor is current times voltage}
\item{If your impulse response is a step, then your IR is an integrator}


\item{cap: $q(t) = C \cdot V(t)$}
\item{intergral of the currente divied by the capacitor}


\item{Mechanical case with springs and masses:
    \subitem{mass: $f(t) = ma(t) = m \stackrel{\cdot}{v}(t) = $ oops } 
}

\item{Spring: $f = kx$
    Equal to k times integral of velocity.
}

\item{
    Inductor = v = L (di/dt)
}

\item{
How about two poles? $H(s) = \frac{1}{s^2}$
}

\item{
Two integrators in series for IR(?)
}

\item{
The integral of the step $u(t)$, which is $t$, more specifically $t \cdot u(t)$. 
}

\item{
3 poles at dc: $H(s) = \frac{1}{s^3}$
}

\item{
Integral at dT, which is $1/2 * t^2$
}

\item{
4 poles at dc: $H(s) = \frac{1}{s^4}$
}

\item{
Integral at dT, which is $1/3! * t^3$
}

\item{
N: $H_n(s) = \frac{1}{s^N} \leftrightarrow h_n(t) = \frac{1}{(N - 1)!}t^{N - 1}$
}

\item{
Polynomial approximation is known for blowing up ....
}

\item{
When you combine integraters, you are getting a linear combination of series integrators in parrallel. 
}

\item{
Taking it farther: feedback of integrators.
}

\item{
Can we analyze it? Lets set up it up a bit more systematically and label the output of the integrators.
}

\item{
outer most x1, then x2, then x3, etc...
}

\item{
x2 of t is equal to x1 of t dot.
}

\item{
look up: state space with arbitrary numerator/denominator. Every transfer function can be realized this way. 
}


\item{
State space/state variable filter
}
\end{itemize}
