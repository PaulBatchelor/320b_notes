\subsection*{Something or other}
Class notes: tuesday, Feb 2.

Misc. class points that I was able to take down

\begin{itemize}

\item{Ringing: if you "pulse" it, does it ring? This means that you'll hear a sinusoidal frequency.}
\paulhint{TODO: look up butterworth wiki page. we are looking at it right now}

\item{The number of poles is always equal to the number of zeros at infinity. }

\item{Repeated poles and repeated zeros are difficult to deal with numerically. It is "numerically ill-conditioned"}

\item{There's a good chart of various orders of butterworth filters on the wiki with maximally flat. 
These are Bode plots.}

\item{Bode plots are very audio-friendly. "log-log".}

\item{When the "Pole breaks"? Along the j omega axis, you pass the imaginary coordinate part of the pole.}

\end{itemize}

\subsection*{Bode Plots}
Our transfer function:

\begin{align*}
H(s) = \frac{1}{s - \sigma_p}
\end{align*}

Sigma is the real part, omega is the imaginary part. 

Set sigma to zero to get j omega axis:

\begin{align*}
H(j \omega ) = \frac{1}{j \omega - \sigma_p}
\end{align*}

% \begin{align*}
% G(\omega) = H(j \omega ) = \frac{1}{j \omega - \sigma_p}
% \end{align*}


\paulhint{See picture taken at 15:28}\\

We do a graphical derivation:\\
\paulhint{See picture taken at 15:30}\\


More generally\\
\paulhint{See picture taken at 15:33}\\

\paulhint{See better picture taken at 15:36}\\

\paulhint{I'm still not getting the visual perception of poles/zeros. TODO: look this up}\\

\paulhint{TODO: look up Q and memorize the eqn. Remember that you know this from a 
musical/applications perspective.}\\

\begin{itemize}
\item{Butterworth filter is notoriously non-ringing.}
\item{A one pole filter cannot ring.}
\end{itemize}

Elliptic vs Butterworth

\paulhint{See picture taken at 16:17. Try to figure out what it means as you don't know it yet}


