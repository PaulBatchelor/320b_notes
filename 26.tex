Class notes: 2-25-16

\subsection*{State Space Filters}

\begin{itemize}
\item{
Digitized transfer fucntion from page 356.
\begin{align*}
H(z) = \frac{1 + 2z^{-1} + 3z^{-2}}{
1 + \frac{1}{2}z^{-1} + \frac{1}{3}z^{-1}}
\end{align*}
}
\item{
The notation:
\begin{align*}
a(n + 1) = Ax + Bu\\
y(n) = Cx + du
\end{align*}
}
\item{
matlab functions that do a lot of stuff for us
}
\end{itemize}
\subsection*{Matlab: State Space Filters}
\begin{itemize}
\item{They give you state face filters. It's a good place to learn.}
\item{\textbf{tf2ss}: transfer function to state space}
\item{Returns arguments $[A, B, C, D]$}
\item{What type of realization state space $A$ is? Controlled cannoical form, you can see all the coefficients in this row.}
\item{Can turn this into a system using the function \textbf{ss}}
\item{sys = ss(A, B, C, D, Td), where Td is the sampling rate(?) because we want to specify 
that it is discrete}
\item{function \textbf{canon} will create a canonical state space 
realization of a particular type.}
\item{We are going to try 'modal' for type}
\item{\textit{Diagonalized state-space form}}
\item{The other state space: \textit{observable}, but it's called 'companion' in the matlab function canon.}
\item{What happens when we look at them individually?}
\item{In the observable, all the coeffecients from our transfer function are in the rightmost column.}
\item{Diagonal form: what you get for A are just eigenvalues for the A matrix}
\item{A collection of eigen vectors multiplied by A gives you that other product (?)}
\item{Linear algebra tools: Jordan form, determinant, etc...}
\item{Diagonalizing a statespace system is very useful, we see lots of things}
\item{The matrix that diagonalizes a system is an eigenvector (?)}
\item{See section \textit{Diagonalizing State Space Modes}}
\end{itemize}
\subsection*{Repeated poles}
\begin{itemize}
\item{You can't do a parrallel sum, when two residues correspond to one pole: need to do 
them in series.}
\end{itemize}
\subsection*{State variable filter LP/BP/HP}
\begin{itemize}
\item{
\begin{align*}
\sim
\end{align*}
}

\begin{tabular}{c | c}
Mode M & Type \\
\hline
0 & lowpass \\
1 & bandpass \\
2 & highpass \\
\end{tabular}

\begin{align*}
\end{align*}

\item{ 
\begin{align*}
H(s) &= \frac{(s / \omega_c)^M}{(s/ \omega_c)^2 + \frac{s/\omega_c}{Q} + 1}  \\
\end{align*}
}

\end{itemize}
\subsection*{Digitization $\frac{1}{s}$}

\begin{itemize}
\item{Backward Euler: $S = \frac{1 - z^{-1}}{T}$}
\item{Forward Euler: $S = \frac{z - 1}{T}$}
\item{BE: $\frac{1}{s} \rightarrow \frac{T}{1 - z^{-1}}$ (integrator)}
\item{FE: $\frac{1}{s} \rightarrow \frac{T}{z - 1}$}
\item{Let's digitize the integrator in a slick way... $\frac{1}{s} \rightarrow \frac{\omega_c}{s}$ }
\item{BE: $\frac{1}{s} \rightarrow \frac{\omega_c}{s} \rightarrow \frac{\omega_c T}{1 - z^{-1}}$}
\item{FE: $\frac{1}{s} \rightarrow \frac{\omega_c}{s} \rightarrow \frac{(\omega_c T)z^{-1}}{1 - z^{-1}}$} 
\item{$\stackrel{\sim}{s} \stackrel{\Delta}{=} \frac{s}{\omega_c} \rightarrow \frac{1}{\stackrel{\sim}{s} = \frac{\omega_c}{s}}$}
\item{back to digitizing... we are going to put this in FAUST
}
\end{itemize}


