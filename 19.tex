\begin{itemize}
\item{Class notes: Tue, Feb 9}
\item{\subsection*{Spectral Centroid}}

Basic idea: spectrum (magnitude or magnitude squared), and you're looking for
the balance point of the spectrum.

\paulhint{chart at 15:06}


With the equation:

\paulhint{eqn at 15:06}

\subsection*{Power Spectral Centroid}
\paulhint{eqn at 15:06}

this eqn has normalization


with L2 norm:

\paulhint{eqn at 15:08}

\item{This allows us to use differentian.}

\item{Normalization in frequency domain is normaliztion in time domain (paricbles thereom?)}

\paulhint{eqn at 15:09}

\item{What is the application for this? Great automatic definition of brightness in a track.}

\item{How to recover the spectral envelope? You use the cepstrum. }

\item{CEPStrum and SPECtrum are related (flip words).}


\item{Pitch adaptive MFCCs are rare/nonexistant. usually a fixed pitch}
\item{16kHz (sometimes 12kHz): standard speech recgonition SR. 8khz: phone SR}
\item{With MFCCs you get the whole envelope, and all the formants, up to 6 resonances with 16kHz}
\item{8kHz: 3, 4 resonances for speech}
\item{We only need as many resonances for intelligibility, in reality they go to infinity}
\item{How do you do this in the time domain?}
\item{What's the differences between windowed cepstrum vs spectral peaks with interpolation?
These are two methods. Cepstrum gets some pyschoacoustic weighting.}
\item{Another contender: LPC - linear predictive coding: a particular kind of spectral envelope that fits an
all pole filter to the spectrum. It is a peak-biased spectral envelope. Even more pyschoacoustic thrust.}
\item{LPC is an all-pole model. (what is an all-pole filter?)}
\item{All these will be covered in 421!}
\end{itemize}
