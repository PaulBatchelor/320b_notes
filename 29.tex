Video notes: vid17.mp4
\subsection*{Mel Frequency Cepstral Coefficients (MFCC)}
\begin{itemize}
\item{These things are very nice features vectors for steady state spectra}
\item{Typically real cepstrum is used: 
\begin{align*}
\stackrel{\sim}{h}(n) = ln \vert H \vert
\end{align*}
}
\item{
The log magnitude spectrum is very close to an auditory spectrum.
}
\item{
Suppose we have a speech signal with formants and formant shaping. We
find the spectral envelope to find the peak frequency of the resonants.
}
\item{
at least 3 formants needed for a vowel. 
}
\item{
This is a nice indicator of steady state spectrum. Spectral envelope on the 
resolution that ignores pitch but captures formant frequency is a indicator
of the timbre of the vowel. 
}
\item{
How do you smooth a log magnitude spectrum to get the spectral envelope? 
You filter it. If you do FFT filtering, what you'd do is take the FFT of the
log magnitude and window that. And that's what the cepstrum does when you
window the cepstrum.
}
\item{
If you go to the cepstral domain (time domain but instead log magnitidue 
was taken), you'll have cepstral coefficents which represent your spectral envelope
initial sequence. Typical number is 12-13 coefficients.  
}
\item{
You get what is called a pitch pulse due to the periodicity of the harmonics.
}
\item{
You low pass filter or "lifter" the spectrum by windowing it before the pitch
pulse. 
}
\item{
\textit{Lifter} is a windowing of the cepstrum.
}
\item{
Cepstral coefficients are linearly related to the spectral envelope on a
dB magnitude scale.
}
\item{
Mel frequency: frequency scale that is more auditory in nature. Taken hold 
as the main oneTaken hold 
}
\item{
Alternative scales: Bark, Equivalent Rectanglular Bandwidth (ERB)
}
\item{
Can be thought of as quasi log scale for frequency. 
}
\end{itemize}
