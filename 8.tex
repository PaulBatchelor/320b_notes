\subsection*{Simplest Electrical LPF}

video: vid6.mp4

Practical lowpass electrical filter is an RC filter

\paulhint{First electrical diagram made}

\paulhint{See 6a.png}

Resistor: made out of carbon composite. $V=RI$ Material such that when you 
put a current across it, the current is proportional to the voltage.

\paulhint{This is the second diagram in the page}

\paulhint{See 6b.png}

Capacitors: charged with voltage. Capacitance %spelling?
how much charge can be held in the plates. $Q = cv$

Current is the dirivative of charge.


\paulhint{This is the third diagram in the page}


Inductor: what is this?
\paulhint{This is the fourth diagram in the page}

RLC circuits
\paulhint{See 6c.png for capacitor and inductor}
Approx: 5:36
Analysis of these filters. (Done more formal than necessary).

\paulhint{This is more diagrams. Worth a rewatch maybe?}

Kirchhoff Loop and Node equations: sum of voltages around a loop is zero.

Loops: 

input loope:

$-v_i + v_r + v_c = 0$ \\
$v_i = v_r + v_c$ \\

output loop:
$-v_c + v_0 = 0$ \\

Node: \\
$i_R - i_c + i_0 = 0$\\
$i_c = i_0 + R_c$

Assume output current ($i_0$) is zero.

\paulhint{I'm missing a lot here. Google Kirchhoff}

$V_i = V_r + V_c = V_R + V_0$ \\

$V_0 = V_i - V_R$ \\

Putting it in the laplace domain:\\

$V_0 = V_i - V_R = V_i - IR$ \\
$= V_i - (V_0 c s) * R$

\paulhint{$V_0 c s = I$}

$V_0[I + R c s] = V$

$H(s) = V_0(s) / V_i(s) = 1 / 1 + RCs$

$ = \frac{1/rc}{s + 1/rc} = \frac{1/\tau}{ s + 1/\tau}$

\josquote{Keep swapping in component definitions until you are ready to find the final solution.}


\paulhint{$\tau \stackrel{\Delta}{=} RC$}

$H(s) 
= \frac{1}{1 + R C S} 
\leftrightarrow 
h(t) = \frac{1}{\tau}e^{-t / \tau}$

\josquote{Be comfortable in both the Z plane and the S plane.}
